%!TEX program = xelatex
\documentclass[UTF8,zihao=5]{ctexart}


\title{{\bfseries 第4次作业}}
\author{周涵宇 2022310984}
\date{}

\usepackage[a4paper]{geometry}
\geometry{left=0.75in,right=0.75in,top=1in,bottom=1in}

\usepackage[
UseMSWordMultipleLineSpacing,
MSWordLineSpacingMultiple=1.5
]{zhlineskip}

\usepackage{fontspec}
\setmainfont{Times New Roman}
% \setmonofont{JetBrains Mono}
\setCJKmainfont{仿宋}[AutoFakeBold=true]
\setCJKsansfont{黑体}[AutoFakeBold=true]

% \usepackage{bm}

\usepackage{amsmath,amsfonts}
\usepackage{array}
\usepackage{float}

\newcommand{\bm}[1]{{\mathbf{#1}}}

\newcommand{\trans}[0]{^\mathrm{T}}
\newcommand{\tran}[1]{#1^\mathrm{T}}
\newcommand{\hermi}[0]{^\mathrm{H}}

\newcommand*{\av}[1]{\left\langle{#1}\right\rangle}

\newcommand*{\avld}[1]{\frac{\overline{D}#1}{Dt}}

\newcommand*{\pd}[2]{\frac{\partial #1}{\partial #2}}

\newcommand*{\pdcd}[3]
{\frac{\partial^2 #1}{\partial #2 \partial #3}}


\begin{document}

\maketitle

\subsection*{4.1(3-1)}

\subsubsection*{(1)}

$$
    x^{(k)}\rightarrow (0, 1, 3)\trans
$$

$$
    \sqrt{k^2+k} - k = \frac{k}{\sqrt{k^2+k} + k}
$$
则由洛必达:
$$
    x^{(k)}\rightarrow (0, 0, \frac{1}{2})\trans
$$

\subsection*{4.2(3-5)}



考虑直接写出迭代矩阵:
$$
    B_J=\begin{bmatrix}
        0                      & -\frac{a_{12}}{a_{11}} \\
        -\frac{a_{21}}{a_{22}} & 0                      \\
    \end{bmatrix},\hspace*{2em}
    B_G=\begin{bmatrix}
        a_{11} & 0      \\
        a_{21} & a_{22} \\
    \end{bmatrix}^{-1}\begin{bmatrix}
        0 & -a_{12} \\
        0 & 0       \\
    \end{bmatrix}
    =
    \begin{bmatrix}
        0 & -\frac{a_{12}}{a_{11}}           \\
        0 & \frac{a_{12}a_{21}}{a_{11}a{22}} \\
    \end{bmatrix}
$$

则
$$
    \rho(B_J)=\sqrt{\left|\frac{a_{12}a_{21}}{a_{11}a{22}}\right|},
    \ \ \ \
    \rho(B_G)=\rho(B_J)^2
$$

\subsubsection*{(1)}

因此$\rho(B_J)<1\Leftrightarrow\rho(B_G)<1$,因此
收敛条件等价。

\subsubsection*{(2)}

渐进收敛率:$\frac{R_J}{R_G}=\frac{-\ln\rho(B_J)}{-\ln\rho(B_G)}=\frac{1}{2}$。

\subsubsection*{(3)}
收敛的充要条件都是:

$$
    \left|\frac{a_{12}a_{21}}{a_{11}a{22}}\right|
    =\frac{1}{a^2}<1\Leftrightarrow
    |a| >1
$$

\subsection*{4.3(3-11)}

首先很明显,如果收敛则为原方程的解。

因此,收敛条件等价于:
$$
    \rho(B)=\rho(I+\alpha A)<1
$$
此处直接计算:
$$
    \rho\left(
    \begin{bmatrix}
            3\alpha + 1 & \alpha      \\
            2\alpha     & 2\alpha + 1 \\
        \end{bmatrix}
    \right) = \max\left\{
    |1+4\alpha|, |1+\alpha|
    \right\}<1
    \Leftrightarrow
    -\frac{1}{2}<\alpha<0
$$

由于$\rho=-1-4\alpha$在$\alpha\in[-0.5,-0.4]$,
且$\rho=1+\alpha$在$\alpha\in[-0.4,0]$,可知其最小值取在
$\alpha=-0.4$处,此时渐进收敛率最大。


\subsection*{4.4(3-12)}
迭代矩阵:

$$
    B=I-\omega A
$$

若$\omega<0$明显有谱半径大于1,不讨论,同时不讨论$\omega=0$,
则其有特征值$1-\omega\lambda_1\leq ...\leq 1-\omega\lambda_n<1$
则$\rho(B)=\max\left\{|1-\omega\lambda_1|,|1-\omega\lambda_n|\right\}=\max\left\{\omega\lambda_1-1,1-\omega\lambda_n\right\}$
因此,$\rho(B)<1\Leftrightarrow  0<\omega \lambda_1<2$
因此有等价于$\omega\in\left(0,\frac{2}{\lambda}\right)$。

由于当$\omega\in\left(0,\frac{2}{\lambda}\right)$,
$\rho(B)$最小值明显取在$\omega\lambda_1-1=1-\omega\lambda_n$处,
因此有此时$\omega=\frac{2}{\lambda_1+\lambda_n}$,这时收敛最快。

\subsection*{4.5(3-17)}

由$\|B_J\|_\infty<1$可知$\forall i,\ \sum_{j,j\neq i}{\left|\frac{a_{ij}}{a_{ii}}\right|}<1$,
也就是说,$|a_{ii}| > \sum_{j,j\neq i}{|a_{ij}|}$,因此$A$是
严格对角占优的。

假设$B_G$有一特征值$\lambda,\ |\lambda| \geq 1$,即有:
$|(D-L)^{-1}U - \lambda I|=0$,因此
$|(D-L)^{-1}||-D+L+U/\lambda|=0$,考虑$(D-L)^{-1}$是非奇异的(对角线
都是非零);且$-D+L+U/\lambda$一定是严格对角占优的,因此非奇异,因此
$|(D-L)^{-1}||-D+L+U/\lambda|\neq 0$,因此矛盾,因此$B_G$的谱半径小于1,
因此G法收敛。

\subsection*{4.6(3-18)}

以下方程都是对称正定的,因此CG收敛。

\subsubsection*{(1)}

取$x^{(0)}=0$,因此$r^{(0)}=p^{(0)}=[0\ -1]\trans$

第0步:
$$
    \alpha_0=\frac{1}{2},\
    x^{(1)} = x^{(0)} + \alpha_0p^{(0)}=[0\ -1/2]\trans,\
    r^{(1)} = [3/2\ 0]\trans,\
    \beta_0 = \frac{9}{4},\
    p^{(1)} = r^{(1)} + \beta_0p^{(0)}=[3/2\ -9/4]\trans
$$

第1步:
$$
    \alpha_1=\frac{2}{3},\
    x^{(2)} = x^{(1)} + \alpha_1p^{(1)}=[1\ -2]\trans,\
    r^{(1)} = [0\ 0]\trans
$$

由于残差是0终止,解为$[1\ -2]\trans$。

\subsubsection*{(2)}

取$x^{(0)}=0$,因此$r^{(0)}=p^{(0)}=[3\ 5\ -5]\trans$

第0步:
$$
    \begin{aligned}
        \alpha_0 & = 0.156914893617021,                                                  \\
        x^{(1)}  & =[0.470744680851064\ 0.784574468085106\ -0.784574468085106]\trans,    \\
        r^{(1)}  & = [-1.236702127659575\ -0.335106382978722\ -1.077127659574468]\trans, \\
        \beta_0  & = 0.047490380262562,                                                  \\
        p^{(1)}  & =[-1.094230986871888\ -0.097654481665911\ -1.314579560887280]\trans   \\
    \end{aligned}
$$

第1步:
$$
    \begin{aligned}
        \alpha_1 & = 0.231099041975764,                                                \\
        x^{(2)}  & =[0.217868948084775\ 0.762006610927475\ -1.088372545207078]\trans,  \\
        r^{(2)}  & = [-0.157495625121525\ 0.209994166828698\ 0.115496791755786]\trans, \\
        \beta_1  & = 0.029351860843840,                                                \\
        p^{(2)}  & =[-0.189613340779206\ 0.207127826072063\ 0.076911435416466]\trans   \\
    \end{aligned}
$$

第2步:
$$
    \begin{aligned}
        \alpha_2 & = 1.149016979445934,                                                            \\
        x^{(3)}  & =[-4.996003610813204e-16\ 0.999999999999999\ -0.999999999999999]\trans,         \\
        r^{(3)}  & = [3.552713678800501e-15\ 5.329070518200751e-15\ -4.440892098500626e-15]\trans, \\
    \end{aligned}
$$
由于迭代次数足够,此时残差达到机器0,实际的解就是$[0\ 1\ -1]\trans$。

\subsection*{4.7(3-19)}
设这一组非零向量是线性相关的,则有一组不都是0的系数$k_i$使得:
$\sum_i{k_ip^{(i)}}=0$,则两边左乘$p^{(m)}{}\trans A$即有:
$\sum_i{k_ip^{(m)}{}\trans A p^{(i)}}=0$,由于互相共轭,有:
${k_mp^{(m)}{}\trans A p^{(m)}}=0$,由于对称正定有:
$p^{(m)}{}\trans A p^{(m)}\neq 0$,则有$k_m=0$,这对所有$m$都成立,
因此与系数不全为0矛盾,因此是线性无关的。

\subsection*{4.8(3-20)}

考虑CG中一步:

$$
    \begin{aligned}
        \varphi(x^{(k+1)}) & =\varphi(x^{(k)} + \alpha p^{(k)})  \\
                           & =
        \frac{1}{2}(Ax^{(k)},x^{(k)}) + \frac{\alpha^2}{2}(Ap^{(k)},p^{(k)})
        +\alpha(x^{(k)}, Ap^{(k)})-\alpha(p^{(k)},b)-(x^{(k)},b) \\
                           & =
        \frac{1}{2}(Ax^{(k)},x^{(k)}) + \frac{\alpha^2}{2}(Ap^{(k)},p^{(k)})
        -\alpha(r^{(k)}, p^{(k)})-(x^{(k)},b)                    \\
    \end{aligned}
$$

这一步内求解关于$\alpha$的最小值问题,明显关于$\alpha$是抛物线,其最小值取在
$\alpha=\frac{(r^{(k)}, p^{(k)})}{(Ap^{(k)},p^{(k)})}$处,
又由于,$(r^{(k+1)},p^{k})=(r^{(k)},p^{(k)}) - \alpha(Ap^{(k),p^{(k)}})=0$,
此时可知,$(r^{(k)},p^{k})=(r^{(k)},r^{(k)})+\beta_{k-1}(r^{(k)},p^{(k-1)})=(r^{(k)},r^{(k)})$,
因此CG法的选取$\alpha_k=\frac{(r^{(k)}, r^{(k)})}{(Ap^{(k)},p^{(k)})}$就是
上述的最小值点,又$\alpha=0$时$\varphi(x^{(k+1)})=\varphi(x^{(k)})$,
因此在最小值点上$\varphi(x^{(k+1)})\leq\varphi(x^{(k)})$。

同时,$r\neq0$时,$\alpha_k\neq0$,因此$0$不是最小值点,因此不等号严格成立。

\subsection*{4.9(4-5)}
考虑$g(x)=x-\lambda f(x)$,考虑$f$导数和$\lambda$范围,
可知$-1<g'(x)<1$。由于其连续性,$[x_a,x_b]$上定积分
可知$g(x_b)-g(x_a)<x_b-x_a$,对于任意$x_a,x_b\in\Re$都成立,
因此这个不动点迭代一定收敛于其已知的不动点$x^*$。

\subsection*{4.10(4-7)}

$$
    \frac{||x_{k+1}-x^*| - |x^*-x_k|| }{|x_{k}-x^*|}\leq\frac{|x_{k+1}-x_k|}{|x_{k}-x^*|}\leq\frac{|x_{k+1}-x^*| + |x^*-x_k| }{|x_{k}-x^*|}
$$

因此
$$
    \left|1-\frac{|x_{k+1}-x^*|}{|x_{k}-x^*|}\right|\leq\frac{|x_{k+1}-x_k|}{|x_{k}-x^*|}\leq 1+\frac{|x_{k+1}-x^*|}{|x_{k}-x^*|}
$$
$$
    \lim_{k\rightarrow\infty}\left|1-\frac{|x_{k+1}-x^*|}{|x_{k}-x^*|}\right|\leq\lim_{k\rightarrow\infty}\frac{|x_{k+1}-x_k|}{|x_{k}-x^*|}\leq 1+\lim_{k\rightarrow\infty}\frac{|x_{k+1}-x^*|}{|x_{k}-x^*|}
$$

考虑$\lim_{k\rightarrow\infty}\frac{|x_{k+1}-x^*|}{|x_{k}-x^*|^p} = C,\ p>1$则有$\lim_{k\rightarrow\infty}\frac{|x_{k+1}-x^*|}{|x_{k}-x^*|}=0$
,根据夹逼定理,则有:

$$
    \lim_{k\rightarrow\infty}\frac{|x_{k+1}-x_k|}{|x_{k}-x^*|}=1
$$

\subsection*{4.11(4-19)}

记
$$
    f(x)=\begin{bmatrix}
        0.7\sin{x_1}+0.2\cos{x_2} \\
        0.7\cos{x_1}-0.2\sin{x_2} \\
    \end{bmatrix}
$$

又容易证明:$|\sin{a}-\sin{b}|\leq|a-b|, |\cos{a}-\cos{b}|\leq|a-b|$,
因此$\|f(x)\|_\infty<0.9\|x\|_\infty$
且当$x\in[-1,1]\times[-1,1],\ f(x)\in[-1,1]\times[-1,1]$,
因此在此范围内有唯一不动点,且不动点迭代收敛。

要求迭代达到误差的无穷范数小于$10^{-3}$,则考虑(以下范数都是无穷范数):
$$
    \|x^{(k+p)}-x^{(k)}\|\leq \|x^{k+1}-x^{k}\|(1+0.9+0.9^2+...+0.9^{p-1})
    =
    \|x^{k+1}-x^{k}\| \frac{1-0.9^p}{1-0.9}
$$
使$p\rightarrow\infty$则:
$$
    \|x^*-x^{(k)}\|\leq
    10\|x^{k+1}-x^{k}\|
$$

因此迭代到$\|x^{k+1}-x^{k}\|<10^{-4}$即可。

初值$x^(0)=[0.5,0.5]\trans$,则有:

$$
    x^{(11)}=[0.526317253520951,\ 0.508007625106778]\trans
$$
以及
$$
    x^{(12)}=[0.526389775735677,\ 0.507976587564812]\trans
$$
因此$\|x^{(12)}-x^{(11)}\| < 0.8\times 10^{-4}$,因此此时满足精度要求。


\subsection*{4.12(4-20)(仅使用牛顿法)}

\subsubsection*{(1)}

$$
    F'(x)=\begin{bmatrix}
        1-0.7\cos{x_1} & 0.2\sin{x_2}   \\
        0.7\sin{x_1}   & 1+0.2\cos{x_2} \\
    \end{bmatrix}
$$

$x^{(0)}=[0.5,\ 0.5]\trans$则

$$
    \begin{aligned}
        F'(x^{(0)}) & =\begin{bmatrix}
            0.385692206676739 & 0.095885107720841 \\
            0.335597877022942 & 1.175516512378075 \\
        \end{bmatrix} \\
        F(x^{(0)})  & =\begin{bmatrix}
            -0.011114389401017 \\-0.018422685602420
        \end{bmatrix} \\
        dx^{(0)}    & =\begin{bmatrix}
            0.026824437769840 \\0.008013891030310
        \end{bmatrix} \\
        x^{(1)}     & =\begin{bmatrix}
            0.526824437769840 \\0.508013891030310
        \end{bmatrix} \\
    \end{aligned}
$$

$$
    \begin{aligned}
        F'(x^{(1)}) & =\begin{bmatrix}
            0.394914350967001 & 0.097288583897780 \\
            0.351953629522860 & 1.174742471778228 \\
        \end{bmatrix} \\
        F(x^{(1)})  & =\begin{bmatrix}
            1.283364687517108e-04 \\2.168258950911439e-04
        \end{bmatrix} \\
        dx^{(1)}    & =\begin{bmatrix}
            -3.017760571481382e-04 \\-9.416082173935467e-05
        \end{bmatrix} \\
        x^{(2)}     & =\begin{bmatrix}
            0.526522661712692 \\0.507919730208571
        \end{bmatrix} \\
    \end{aligned}
$$

由于$F(x^{(2)})=[1.679789332631465e-08,\ 2.712255181558376e-08]\trans $,
且$\|F'(x^{(2)})^{-1}\|_\infty=2.9611$,可知此时误差也在$10^{-8}$量级。

\subsubsection*{(2)}

$$
    F(x)=\begin{bmatrix}
        x_1^2+x_2^2-4 \\
        x_1^2-x_2^2-1 \\
    \end{bmatrix},\
    F'(x)=\begin{bmatrix}
        2x_1 & 2x_2  \\
        2x_1 & -2x_2 \\
    \end{bmatrix}
$$  

$$
    \begin{aligned}
        F'(x^{(0)}) & =\begin{bmatrix}
            3.200000000000000&2.400000000000000\\
            3.200000000000000&-2.400000000000000
        \end{bmatrix} \\
        F(x^{(0)})  & =\begin{bmatrix}
            0\\0.120000000000001
        \end{bmatrix} \\
        dx^{(0)}    & =\begin{bmatrix}
            -0.018750000000000\\0.025000000000000
        \end{bmatrix} \\
        x^{(1)}     & =\begin{bmatrix}
            1.581250000000000\\1.225000000000000
        \end{bmatrix} \\
    \end{aligned}
$$

$$
    \begin{aligned}
        F'(x^{(1)}) & =\begin{bmatrix}
            3.162500000000000&2.450000000000000\\
            3.162500000000000&-2.450000000000000
        \end{bmatrix} \\
        F(x^{(1)})  & =\begin{bmatrix}
            9.765625000000000e-04\\-2.734375000001954e-04
        \end{bmatrix} \\
        dx^{(1)}    & =\begin{bmatrix}
            -1.111660079051075e-04\\-2.551020408163664e-04
        \end{bmatrix} \\
        x^{(2)}     & =\begin{bmatrix}
            1.581138833992095\\1.224744897959184
        \end{bmatrix} \\
    \end{aligned}
$$

由于$F(x^{(2)})=[-3.907905283597240e-09,\ -2.656759445734503e-08]\trans $,
且$\|F'(x^{(2)})^{-1}\|_\infty=0.4082$,可知此时误差也在$10^{-8}$量级。

\end{document}