%!TEX program = xelatex
\documentclass[UTF8,zihao=5]{ctexart}


\title{{\bfseries 第4次作业}}
\author{周涵宇 2022310984}
\date{}

\usepackage[a4paper]{geometry}
\geometry{left=0.75in,right=0.75in,top=1in,bottom=1in}

\usepackage[
UseMSWordMultipleLineSpacing,
MSWordLineSpacingMultiple=1.5
]{zhlineskip}

\usepackage{fontspec}
\setmainfont{Times New Roman}
% \setmonofont{JetBrains Mono}
\setCJKmainfont{仿宋}[AutoFakeBold=true]
\setCJKsansfont{黑体}[AutoFakeBold=true]

% \usepackage{bm}

\usepackage{amsmath,amsfonts}
\usepackage{array}
\usepackage{float}

\newcommand{\bm}[1]{{\mathbf{#1}}}

\newcommand{\trans}[0]{^\mathrm{T}}
\newcommand{\tran}[1]{#1^\mathrm{T}}
\newcommand{\hermi}[0]{^\mathrm{H}}

\newcommand*{\av}[1]{\left\langle{#1}\right\rangle}

\newcommand*{\avld}[1]{\frac{\overline{D}#1}{Dt}}

\newcommand*{\pd}[2]{\frac{\partial #1}{\partial #2}}

\newcommand*{\pdcd}[3]
{\frac{\partial^2 #1}{\partial #2 \partial #3}}


\begin{document}

\maketitle

\subsection*{4.1(3-1)}

\subsubsection*{(1)}

$$
    x^{(k)}\rightarrow (0, 1, 3)\trans
$$

$$
    \sqrt{k^2+k} - k = \frac{k}{\sqrt{k^2+k} + k}
$$
则由洛必达:
$$
    x^{(k)}\rightarrow (0, 0, \frac{1}{2})\trans
$$

\subsection*{4.2(3-5)}



考虑直接写出迭代矩阵:
$$
    B_J=\begin{bmatrix}
        0                      & -\frac{a_{12}}{a_{11}} \\
        -\frac{a_{21}}{a_{22}} & 0                      \\
    \end{bmatrix},\hspace*{2em}
    B_G=\begin{bmatrix}
        a_{11} & 0      \\
        a_{21} & a_{22} \\
    \end{bmatrix}^{-1}\begin{bmatrix}
        0 & -a_{12} \\
        0 & 0       \\
    \end{bmatrix}
    =
    \begin{bmatrix}
        0 & -\frac{a_{12}}{a_{11}}           \\
        0 & \frac{a_{12}a_{21}}{a_{11}a{22}} \\
    \end{bmatrix}
$$

则
$$
    \rho(B_J)=\sqrt{\left|\frac{a_{12}a_{21}}{a_{11}a{22}}\right|},
    \ \ \ \
    \rho(B_G)=\rho(B_J)^2
$$

\subsubsection*{(1)}

因此$\rho(B_J)<1\Leftrightarrow\rho(B_G)<1$,因此
收敛条件等价。

\subsubsection*{(2)}

渐进收敛率:$\frac{R_J}{R_G}=\frac{-\ln\rho(B_J)}{-\ln\rho(B_G)}=\frac{1}{2}$。

\subsubsection*{(3)}
收敛的充要条件都是:

$$
    \left|\frac{a_{12}a_{21}}{a_{11}a{22}}\right|
    =\frac{1}{a^2}<1\Leftrightarrow
    |a| >1
$$

\subsection*{4.3(3.11)}

首先很明显,如果收敛则为原方程的解。

因此,收敛条件等价于:
$$
    \rho(B)=\rho(I+\alpha A)<1
$$
此处直接计算:
$$
    \rho\left(
    \begin{bmatrix}
            3\alpha + 1 & \alpha      \\
            2\alpha     & 2\alpha + 1 \\
        \end{bmatrix}
    \right) = \max\left\{
    |1+4\alpha|, |1+\alpha|
    \right\}<1
    \Leftrightarrow
    -\frac{1}{2}<\alpha<0
$$

由于$\rho=-1-4\alpha$在$\alpha\in[-0.5,-0.4]$,
且$\rho=1+\alpha$在$\alpha\in[-0.4,0]$,可知其最小值取在
$\alpha=-0.4$处,此时渐进收敛率最大。   


\subsection*{4.4(3.12)}
迭代矩阵:

$$
B=I-\omega A
$$

若$\omega<0$明显有谱半径大于1,不讨论,同时不讨论$\omega=0$,
则其有特征值$1-\omega\lambda_1\leq ...\leq 1-\omega\lambda_n<1$
则$\rho(B)=\max\left\{|1-\omega\lambda_1|,|1-\omega\lambda_n|\right\}=\max\left\{\omega\lambda_1-1,1-\omega\lambda_n\right\}$
因此,$\rho(B)<1\Leftrightarrow  0<\omega \lambda_1<2$
因此有等价于$\omega\in\left(0,\frac{2}{\lambda}\right)$。

由于当$\omega\in\left(0,\frac{2}{\lambda}\right)$,
$\rho(B)$最小值明显取在$\omega\lambda_1-1=1-\omega\lambda_n$处,
因此有此时$\omega=\frac{2}{\lambda_1+\lambda_n}$,这时收敛最快。







\end{document}