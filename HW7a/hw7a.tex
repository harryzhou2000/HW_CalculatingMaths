%!TEX program = xelatex
\documentclass[UTF8,zihao=5]{ctexart}


\title{{\bfseries 第7次作业}}
\author{周涵宇 2022310984}
\date{}

\usepackage[a4paper]{geometry}
\geometry{left=0.75in,right=0.75in,top=1in,bottom=1in}

\usepackage[
UseMSWordMultipleLineSpacing,
MSWordLineSpacingMultiple=1.5
]{zhlineskip}

\usepackage{fontspec}
\setmainfont{Times New Roman}
% \setmonofont{JetBrains Mono}
\setCJKmainfont{仿宋}[AutoFakeBold=true]
\setCJKsansfont{黑体}[AutoFakeBold=true]

% \usepackage{bm}

\usepackage{amsmath,amsfonts}
\usepackage{array}
\usepackage{float}

\newcommand{\bm}[1]{{\mathbf{#1}}}

\newcommand{\trans}[0]{^\mathrm{T}}
\newcommand{\tran}[1]{#1^\mathrm{T}}
\newcommand{\hermi}[0]{^\mathrm{H}}

\newcommand*{\av}[1]{\left\langle{#1}\right\rangle}

\newcommand*{\avld}[1]{\frac{\overline{D}#1}{Dt}}

\newcommand*{\pd}[2]{\frac{\partial #1}{\partial #2}}

\newcommand*{\pdcd}[3]
{\frac{\partial^2 #1}{\partial #2 \partial #3}}


\begin{document}

\maketitle

\subsection*{7.1(7-1)}

从自然基开始对内积$(f,g)=\int_{-1}^{1}{\rho fg dx}$正交化:

有$P_0=1$,$Q_1=x,\ (Q_1,P_0)=0,\ (P_0,P_0)=2$,因此有
$P_1=x,\ (P_1,P_1)=2/3$;取$Q_2=x^2$,则$(Q_2,P_0)=2/3, (Q_2,P_1)=0$,
因此有$P_2=Q_2-P_0\frac{(Q_2,P_0)}{(P_0,P_0)}-P_1\frac{(Q_2,P_1)}{(P_1,P_1)}=x^2-1/3$,
取$Q_3=x^3$,则$(Q_3,P_0)=(Q_3,P_2)=0, (Q_3,P_1)=5/2$,类似有$P_3=x^3-(3/5)x$。

取首一多项式为结果则有:
$$
    P_1=x,\ P_2=x^2-\frac{1}{3},\ P_3=x^3-\frac{3}{5}x
$$

\subsection*{7.2(7-2)}

$Q_1=x,\ (Q_1,L_0)=1,\ (L_0,L_0)=1$,则可知$L_1^*=Q_1-L_0\frac{(Q_1,L_0)}{(L_0,L_0)}=x-1$。
为满足$(L_1,L_1)=1!,\ L_1=1-x$

由递推公式,$L_2=(3-x)(x-1)-1=2-4x+x^2$,$L_3=(5-x)(2-4x+x^2)-4(1-x)=6-18x+9x^2-x^3$。

\subsection*{7.3(7-4)}

零点:$x_1=\cos{\frac{\pi}{6}}=\frac{\sqrt{3}}{2},\ x_2=0,\ x_3=-x_1=-\frac{\sqrt{3}}{2}$

Newton均差表为:
\begin{table}[H]
    \centering
    \begin{tabular}{c|ccc}
        x                                                                                                                                                \\
        \hline
        $-\frac{\sqrt{3}}{2}$ & $\exp(-\frac{\sqrt{3}}{2})$                                                                                              \\
        0                     & 1                           & $\frac{2-2\exp(-\frac{\sqrt{3}}{2})}{\sqrt{3}}$                                            \\
        $\frac{\sqrt{3}}{2}$  & $\exp(\frac{\sqrt{3}}{2})$  & $\frac{2\exp(\frac{\sqrt{3}}{2})-2}{\sqrt{3}}$  & $4\frac{\cosh(\frac{\sqrt{3}}{2})-1}{3}$ \\
    \end{tabular}
\end{table}

因此插值为:
$$
    N_2=\exp(-\frac{\sqrt{3}}{2}) + (x+\frac{\sqrt{3}}{2})\frac{2-2\exp(-\frac{\sqrt{3}}{2})}{\sqrt{3}}
    +x(x+\frac{\sqrt{3}}{2})\left[\frac{4}{3}(\cosh(\frac{\sqrt{3}}{2})-1)\right]
$$

\subsection*{7.4(7-7)}

第一题已经计算了Legendre多项式$P_0,P_1,P_2$,
则$P_1^*=\frac{(P_0,f)}{(P_0,P_0)}P_0+\frac{(P_1,f)}{(P_1,P_1)}P_1=\sinh(1)+x(3e^{-1})$,
$P_2^*=P_1^*+\frac{(P_2,f)}{(P_2,P_2)}P_2=\sinh(1)+x(3e^{-1})+(x^2-1/3)\frac{15}{4}e^{-1}(e^2-7)$

数值地:
$$
    P_1^*\approx 1.1752+1.1036x,\ P_2^*\approx 0.9963+1.1036x+0.5367x^2
$$

\subsection*{7.5(7-10)}
第一题已经计算了Legendre多项式$P_0,P_1,P_2,P_3$,则:
$$
    \frac{(P_0,f)}{(P_0,P_0)}=0,\
    \frac{(P_1,f)}{(P_1,P_1)}=\frac{12}{\pi^2},\
    \frac{(P_2,f)}{(P_2,P_2)}=0,\
    \frac{(P_3,f)}{(P_3,P_3)}=\frac{420(p^2-10)}{\pi^4}
$$

因此逼近多项式
$$
    P_3^*=\frac{12}{\pi^2}x+\frac{420(p^2-10)}{\pi^4}(x^3-\frac{3}{5}x)
    \approx 1.5532x -0.5622x^3
$$

\subsection*{7.6(7-11)}

Maclaurin级数$x=0$处给出:
$$
    \ln(x+1)=0+x-\frac{x^2}{2}+\frac{x^3}{3}-\frac{x^4}{4}+\frac{x^5}{5}+\dots
$$
$R_{3,2}=\frac{P_3}{Q_2}$中设:
$$
    Q_2=1+q_1x+q_2x^2,\ P_2=p_0+p_1x+p_2x^2+p_3x^3
$$

则根据条件:
$$
    (x-\frac{x^2}{2}+\frac{x^3}{3}-\frac{x^4}{4}+\frac{x^5}{5}+\dots)(1+q_1x+q_2x^2)
    -
    (p_0+p_1x+p_2x^2+p_3x^3)=c_{6}x^6+\dots
$$

展开后得:
$$
    \begin{aligned}
        x^5 & :\ \frac{1}{5}-\frac{q_1}{4}+\frac{q_2}{3}=0  \\
        x^4 & :\ -\frac{1}{4}+\frac{q_1}{3}-\frac{q_2}{2}=0 \\
    \end{aligned}
$$
以及:
$$
    \begin{aligned}
        x^3 & :\ \frac{1}{3}-\frac{q_1}{2}+\frac{q_2}{1}=p_3 \\
        x^2 & :\ -\frac{1}{2}+\frac{q_1}{1}=p_2              \\
        x^1 & :\ \frac{1}{1}=p_1                             \\
        x^0 & :\ 0=p_0                                       \\
    \end{aligned}
$$

前两个方程解出:$q_1=\frac{6}{5},\ q_2=\frac{3}{10}$
因此得到:$p_0=0,\ p_1=1,\ p_2=\frac{7}{10},\ p_3=\frac{1}{30}$
则:

$$
    R_{3,2}=\frac{x+\frac{7}{10}x^2+\frac{1}{30}x^3}
    {1+\frac{6}{5}x+\frac{3}{10}x^2}
$$

\subsection*{7.7(7-17)}

是线性拟合,对于最小二乘拟合,有线性最小二乘问题:
$$
    y_i=a\sin{\pi x_i}+b\cos{\pi x_i}
$$
则求:
$$
    argmin\left\|\begin{bmatrix}
        0  & -1 \\
        -1 & 0  \\
        0  & 1  \\
        1  & 0  \\
        0  & -1
    \end{bmatrix}\begin{bmatrix}
        a \\
        b
    \end{bmatrix}-\begin{bmatrix}
        -1 \\0\\1\\2\\1
    \end{bmatrix}
    \right\|_2^2
$$
求解法方程:
$$
    \begin{bmatrix}
        2 & 0 \\ 0& 3
    \end{bmatrix}
    \begin{bmatrix}
        a \\
        b
    \end{bmatrix}=\begin{bmatrix}
        2\\1
    \end{bmatrix}
$$
可得
$$
a=1,\ b=\frac{1}{3}
$$
 

\end{document}