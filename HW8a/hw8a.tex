%!TEX program = xelatex
\documentclass[UTF8,zihao=5]{ctexart}


\title{{\bfseries 第8次作业}}
\author{周涵宇 2022310984}
\date{}

\usepackage[a4paper]{geometry}
\geometry{left=0.75in,right=0.75in,top=1in,bottom=1in}

\usepackage[
UseMSWordMultipleLineSpacing,
MSWordLineSpacingMultiple=1.5
]{zhlineskip}

\usepackage{fontspec}
\setmainfont{Times New Roman}
% \setmonofont{JetBrains Mono}
\setCJKmainfont{仿宋}[AutoFakeBold=true]
\setCJKsansfont{黑体}[AutoFakeBold=true]

% \usepackage{bm}

\usepackage{amsmath,amsfonts}
\usepackage{array}
\usepackage{float}

\newcommand{\bm}[1]{{\mathbf{#1}}}

\newcommand{\trans}[0]{^\mathrm{T}}
\newcommand{\tran}[1]{#1^\mathrm{T}}
\newcommand{\hermi}[0]{^\mathrm{H}}

\newcommand*{\av}[1]{\left\langle{#1}\right\rangle}

\newcommand*{\avld}[1]{\frac{\overline{D}#1}{Dt}}

\newcommand*{\pd}[2]{\frac{\partial #1}{\partial #2}}

\newcommand*{\pdcd}[3]
{\frac{\partial^2 #1}{\partial #2 \partial #3}}


\begin{document}

\maketitle

\subsection*{8.1(8-1)}

\subsubsection*{(1)}

验证:
$$
    \begin{aligned}
        \int_0^1{1dx}=1             & =\frac{1}{4}\times 1 + \frac{3}{4}\times 1               \\
        \int_0^1{xdx}=\frac{1}{2}   & =\frac{1}{4}\times 0 + \frac{3}{4}\times \frac{2}{3}     \\
        \int_0^1{x^2dx}=\frac{1}{3} & =\frac{1}{4}\times 0 + \frac{3}{4}\times \frac{4}{9}     \\
        \int_0^1{x^3dx}=\frac{1}{4} & \neq\frac{1}{4}\times 0 + \frac{3}{4}\times \frac{8}{27} \\
    \end{aligned}
$$
因此为2次代数精度。

\subsubsection*{(2)}

在区间$[0,1]$验证,根据平移和齐次性$[0,1]$的代数精度就是公式的代数精度:
$$
    \begin{aligned}
        \int_0^1{1dx}= 1            & =\frac{1}{2}(1+1) - \frac{1}{12}(0 - 0)    \\
        \int_0^1{xdx}= \frac{1}{2}  & =\frac{1}{2}(0+1) - \frac{1}{12}(1 - 1)    \\
        \int_0^1{x^2dx}=\frac{1}{3} & =\frac{1}{2}(0+1) - \frac{1}{12}(2 - 0)    \\
        \int_0^1{x^3dx}=\frac{1}{4} & =\frac{1}{2}(0+1) - \frac{1}{12}(3 - 0)    \\
        \int_0^1{x^4dx}=\frac{1}{5} & \neq\frac{1}{2}(0+1) - \frac{1}{12}(4 - 0) \\
    \end{aligned}
$$
因此为3次代数精度。

\subsection*{8.2(8-2)}

\subsubsection*{(1)}

代入$1,x,x^2$有约束:
$$
    \left\{
    \begin{array}{rl}
        C_0 + C_1 + C_2 & = 2          \\
        C_1 + 2C_2      & =2           \\
        C_1 + 4C_2      & =\frac{8}{3} \\
    \end{array}
    \right.
$$
线性方程组,解得$C_0=\frac{1}{3},\ C_1=\frac{4}{3},\ C_2=\frac{1}{3}$,
代入$x^3,x^4$,可知:
$$
    \begin{aligned}
        C_1+8C_2  & =\frac{16}{4}    \\
        C_1+16C_2 & \neq\frac{32}{5}
    \end{aligned}
$$
因此有3次代数精度。

代入$1,x,x^2$有约束:
$$
    \left\{
    \begin{array}{rl}
        \frac{1}{2} + C_1           & = 1          \\
        \frac{1}{2}x_0 + C_1x_1     & =\frac{1}{2} \\
        \frac{1}{2}x_0^2 + C_1x_1^2 & =\frac{1}{3} \\
    \end{array}
    \right.
$$
因此取$x_0<x_1$解得
$C_1=\frac{1}{2},\ x_0=\frac{1}{2}-\frac{\sqrt{3}}{6},\ x_0=\frac{1}{2}+\frac{\sqrt{3}}{6}$

代入$(x-\frac{1}{2})^3,(x-\frac{1}{2})^4$
$$
    \begin{aligned}
        \frac{1}{2}\left[\left(
            -\frac{\sqrt{3}}{6}
            \right)^3+
            \left(
            \frac{\sqrt{3}}{6}
        \right)^3\right]               & =0                \\
        \frac{1}{2}\left[\left(
            -\frac{\sqrt{3}}{6}
            \right)^4+
            \left(
            \frac{\sqrt{3}}{6}
        \right)^4\right] =\frac{1}{72} & \neq \frac{1}{80} \\
    \end{aligned}
$$
因此有3次代数精度,符合Gauss型求积的情况。


\subsection*{8.3(8-5)}
有2次代数精度。
则代入$1,x,x^2$:
$$
    \left\{
    \begin{array}{rl}
        C_0+C_1 & = 1          \\
        B_0+C_1 & =\frac{1}{2} \\
        C_1     & =\frac{1}{3} \\
    \end{array}
    \right.
$$
解得$C_0=\frac{2}{3},\ C_1=\frac{1}{3},\ B_0=\frac{1}{6}$。
考虑这是一个节点Newton插值,节点为$0,0,1$,则直接应用余项公式:
$$
    E(f)=\int_0^1{
    R(x)dx
    }=\int_0^1{
    \frac{1}{3!}f'''(\xi(x))\omega_3(x)dx
    }
$$
其中$\xi(x)\in(0,1)$,由于$\omega_3(x)=x^2(x-1)$不变号,
根据积分中值定理,有:
$$
    E(f)=\frac{1}{3!}f'''(\eta)\int_0^1{
    \omega_3(x)dx
    }
    =-\frac{1}{72}f'''(\eta)
$$
其中$\eta\in(0,1)$,则可知$k=-\frac{1}{72}$。




\subsection*{8.4(8-8)}
为权是$\sqrt{x}$的Gauss积分,
构建正交基。

$P_0=1$,则根据
$$
    \begin{aligned}
        \int_0^1{\sqrt{x}P_0x dx}  & = \frac{2}{5} \\
        \int_0^1{\sqrt{x}P_0^2 dx} & = \frac{2}{3} \\
    \end{aligned}
$$
则取$P_1=x-\frac{3}{5}$。接下来根据
$$
    \begin{aligned}
        \int_0^1{\sqrt{x}P_1P_1 dx} & = \frac{8}{175}  \\
        \int_0^1{\sqrt{x}P_1x^2 dx} & = \frac{16}{315} \\
        \int_0^1{\sqrt{x}P_0x^2 dx} & = \frac{2}{7}    \\
    \end{aligned}
$$
计算得到$P_2=x^2-\frac{10}{9}x+\frac{5}{21}$。
Gauss积分的节点取$P_2$的根,则为
$x_0,x_1=\frac{5}{9}\pm\frac{2\sqrt{70}}{63}$。
双精度结果为:
$$
    x_0=\frac{5}{9}-\frac{2\sqrt{70}}{63}\approx0.289949197925690,\
    x_1=\frac{5}{9}+\frac{2\sqrt{70}}{63}\approx0.821161913185421
$$
代入$x_0,x_1$求系数:
$$
    \left\{
    \begin{array}{rl}
        A_0+A_1       & = \frac{2}{3} \\
        A_0x_0+A_1x_1 & =\frac{2}{5}  \\
    \end{array}
    \right.
$$
解得:
$$
    A_0=\frac{1}{3}-\frac{\sqrt{70}}{150}\approx 0.277555998231062,\ 
    A_1=\frac{1}{3}+\frac{\sqrt{70}}{150}\approx 0.389110668435605
$$
验证可得其有三次代数精度。






\subsection*{8.5(8-10)}
选取代数精度为3的2积分点Gauss-Chebyshev积分公式。
积分点为2次Chebyshev多项式的零点,即$\cos{\frac{\pi}{4}}$和
$\cos{\frac{3\pi}{4}}$,权都是$\pi/2$。则:
$$
\int_{-1}^1{\frac{x^2}{\sqrt{1-x^2}}dx}
=
\frac{\pi}{2}\left(
    (1/\sqrt{2})^2+(-1/\sqrt{2})^2
\right)
=\frac{\pi}{2}
$$

\subsection*{8.6(8-18)}
% 两者都是插值型求导公式,在$\{x_0-h,x_0+3h\}$
% 以及$\{x_0,x_0+h,x_0+2h\}$上进行多项式插值再求导即得。

\subsubsection*{(1)}
$$
\begin{aligned}
    f(x_0+3h)&=f(x_0)+3hf'(x_0)+\frac{9}{2}h^2f''(\xi)\\
    f(x_0-h) &=f(x_0)-hf'(x_0)+\frac{1}{2}h^2f''(\eta)
\end{aligned}
$$
$\xi\in(x_0,x_0+3h),\eta\in(x_0-h,x_0)$,则
$$
\frac{f(x_0+3h)-f(x_0-h)}{4h}
=f'(x_0)+h\left(
    \frac{9}{8}f''(\xi)-\frac{1}{8}f''(\eta)
\right)
=f'(x_0)+hf''(\zeta)
$$
则余项是$-hf''(\zeta)$,其中$\zeta\in(x_0-h,x_0+3h)$

\subsection*{(2)}
Taylor展开法进行分析。

$$
\begin{aligned}
    f(x_0+h)&=f(x_0)+hf'(x_0)+\frac{1}{2}h^2f''(x_0)+\frac{1}{6}h^3f'''(\xi)\\
    f(x_0+2h)&=f(x_0)+2hf'(x_0)+\frac{4}{2}h^2f''(x_0)+\frac{8}{6}h^3f'''(\eta)\\
\end{aligned}
$$
$\xi\in(x_0,x_0+h),\eta\in(x_0,x_0+2h)$,则
$$
\frac{4f(x_0+h)-3f(x_0)-f(x_0+2h)}{2h}
=f'(x_0)+h^2\left(
    \frac{4}{12}f'''(\xi)-\frac{8}{12}f'''(\eta)
\right)
=f'(x_0)-h^2\frac{f'''(\zeta)}{3}
$$
则余项是$h^2\frac{f'''(\zeta)}{3}$,其中$\zeta\in(x_0,x_0+2h)$

\subsection*{8.7(8-19)}

采用一阶中心差分:$f'(1)\approx \frac{f(1+h)-f(1-h)}{2h}=G_1(h)$

取$h=0.1$,则有
$$
G(h)=2.722814563947418,\ 
G(h/2)=2.719414587473179,\ 
G(h/4)=2.718564991664882
$$
可见误差是2阶收敛的,
根据$f'(1) - G(h) = \alpha_1h^2 + \alpha_2h^4...$,
采用Richardson外推:

$$
G_2(h)=\frac{4^1G(h/2)-G(h)}{4^1-1}=2.718281261981766,\ 
G_2(h/2)=\frac{4^1G(h/4)-G(h/2)}{4^1-1}=2.718281793062117
$$
可见误差是4阶收敛的,进一步采用外推:

$$
G_3(h)=\frac{4^2G_2(h/2)-G_2(h)}{4^2-1}=2.718281828467474
$$
为数值微分结果,有11位有效数字。
$G_3(h)$相当于一个误差为$O(h^6)$数值微分公式。



\end{document}