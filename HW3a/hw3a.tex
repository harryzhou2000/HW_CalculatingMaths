%!TEX program = xelatex
\documentclass[UTF8,zihao=5]{ctexart}


\title{{\bfseries 第3次作业}}
\author{周涵宇 2022310984}
\date{}

\usepackage[a4paper]{geometry}
\geometry{left=0.75in,right=0.75in,top=1in,bottom=1in}

\usepackage[
UseMSWordMultipleLineSpacing,
MSWordLineSpacingMultiple=1.5
]{zhlineskip}

\usepackage{fontspec}
\setmainfont{Times New Roman}
% \setmonofont{JetBrains Mono}
\setCJKmainfont{仿宋}[AutoFakeBold=true]
\setCJKsansfont{黑体}[AutoFakeBold=true]

% \usepackage{bm}

\usepackage{amsmath,amsfonts}
\usepackage{array}
\usepackage{float}

\newcommand{\bm}[1]{{\mathbf{#1}}}

\newcommand{\trans}[0]{^\mathrm{T}}
\newcommand{\tran}[1]{#1^\mathrm{T}}
\newcommand{\hermi}[0]{^\mathrm{H}}

\newcommand*{\av}[1]{\left\langle{#1}\right\rangle}

\newcommand*{\avld}[1]{\frac{\overline{D}#1}{Dt}}

\newcommand*{\pd}[2]{\frac{\partial #1}{\partial #2}}

\newcommand*{\pdcd}[3]
{\frac{\partial^2 #1}{\partial #2 \partial #3}}


\begin{document}

\maketitle

\subsection*{3.1(2-3)}

无需选主元直接分解:
$$
    \begin{bmatrix}
        6  & 2 & 1  & -1 \\
        2  & 4 & 1  & 0  \\
        1  & 1 & 4  & -1 \\
        -1 & 0 & -1 & 3  \\
    \end{bmatrix}
    =
    \left[\begin{array}{cccc} 1 & 0 & 0 & 0\\ \frac{1}{3} & 1 & 0 & 0\\ \frac{1}{6} & \frac{1}{5} & 1 & 0\\ -\frac{1}{6} & \frac{1}{10} & -\frac{9}{37} & 1 \end{array}\right]
    \left[\begin{array}{cccc} 6 & 2 & 1 & -1\\ 0 & \frac{10}{3} & \frac{2}{3} & \frac{1}{3}\\ 0 & 0 & \frac{37}{10} & -\frac{9}{10}\\ 0 & 0 & 0 & \frac{191}{74} \end{array}\right]
$$

因此有:
$$
    L^{-1}b=
    \left[\begin{array}{c} 6\\ -1\\ \frac{21}{5}\\ \frac{527}{74} \end{array}\right]
$$
接着:
$$
    U^{-1}L^{-1}b=
    \left[\begin{array}{c} \frac{281}{191}\\ -\frac{179}{191}\\ \frac{345}{191}\\ \frac{527}{191} \end{array}\right]
$$
是解。同时$\det{A}=\det{U}=191$。

\subsection*{3.2(2-8)}

对称正定,有分解$A=LL\trans$:
$$
    L=
    \left[\begin{array}{ccc} 4 & 0 & 0\\ 1 & 2 & 0\\ 2 & -3 & 3 \end{array}\right]
$$

有
$$
    L^{-1}b=
    \left[\begin{array}{c} -1\\ 2\\ 6 \end{array}\right],
    (L\trans)^{-1}L^{-1}b=
    \left[\begin{array}{c} -\frac{9}{4}\\ 4\\ 2 \end{array}\right]
$$

\subsection*{3.3(2-11)}
计算:
$l_1=\sqrt{b_1}, m_2 = \frac{a_2}{\sqrt{b_1}}$,第1列算出。
考虑第$i<n$列:$l_i=\sqrt{b_i-m_i^2}, m_{i+1}=\frac{b_{i+1}}{l_i}$,
最后一列:$l_n=\sqrt{b_n-m_n^2}$。

\subsection*{3.4(2-12)}

首先,$\|L\|_2^2=\rho(LL\trans)=\rho(A)$,$\|A\|_2^2=\rho(AA\trans)=\rho(A^2)$。
同时由于$A$一定有特征分解$A=VDV\trans$,则$A^2$特征分解就是$A^2=VD^2V\trans$,则可知
$\rho(A^2)=\rho(A)^2$,因此$\|L\|_2^4=\rho(A)^2=\rho(A^2)=\|A\|_2^2$,因此得证。

根据以上递推公式即可得到$m_i,l_i$。

\subsection*{3.5(2-15)}

考虑$A^{-1}=\left[\begin{array}{cc} \frac{1}{6} & -\frac{1}{6}\\ \frac{5}{6} & \frac{1}{6} \end{array}\right]$
则$\mathrm{cond}_\infty(A)=\|A\|_\infty\|A^{-1}\|_\infty=6$。

考虑$B^{-1}=\left[\begin{array}{ccc} \frac{3}{4} & \frac{1}{2} & \frac{1}{4}\\ \frac{1}{2} & 1 & \frac{1}{2}\\ \frac{1}{4} & \frac{1}{2} & \frac{3}{4} \end{array}\right]$
则$\mathrm{cond}_2(B)=\|B\|_2\|B^{-1}\|_2=3+2\sqrt{2}\approx5.8$。

\subsection*{3.6(2-18)}

因为$\|A^{-1}\delta A\|\leq\|A^{-1}\|\|\delta A\|< 1$,
若设$I+A^{-1}\delta A$奇异,则$A^{-1}\delta A$有特征值-1,则其谱半径不小于1,则其矩阵范数不小于1,
则是矛盾的。因此$I+A^{-1}\delta A$非奇异,因此$A+\delta A$非奇异。

考虑:
$$
    \begin{aligned}
        1 = \|(I+A^{-1}\delta A)(I+A^{-1}\delta A)^{-1}\| =&
        \|(I+A^{-1}\delta A)^{-1} + A^{-1}\delta A (I+A^{-1}\delta A)^{-1}\|\\
        \geq&
        \|(I+A^{-1}\delta A)^{-1}\| - \| A^{-1}\delta A (I+A^{-1}\delta A)^{-1}\|\\
        \geq&
        \|(I+A^{-1}\delta A)^{-1}\| - \| A^{-1}\delta A\|\| (I+A^{-1}\delta A)^{-1}\|\\
        =&
        (1 - \| A^{-1}\delta A\|)\| (I+A^{-1}\delta A)^{-1}\|
    \end{aligned}
$$
由于$(1 - \| A^{-1}\delta A\|)>0$,即有:
$$
\| (I+A^{-1}\delta A)^-1\|\leq \frac{1}{1 - \| A^{-1}\delta A\|}
$$
因此

\begin{equation}
    \| (A+\delta A)^{-1} \|\leq \frac{\|A^{-1}\|}{1 - \| A^{-1}\delta A\|}
    \leq \frac{\|A^{-1}\|}{1 - \| A^{-1}\|\|\delta A\|}
\end{equation}

因此:
\begin{equation}
    \begin{aligned}
        \frac{\|A^{-1} - (A+\delta A)^{-1}\|}{\|A^{-1}\|} = &
        \frac{\|A^{-1}(A+\delta A)^{-1}\delta A\|}{\|A^{-1}\|} \leq
        \|(A+\delta A)^{-1}\|\|\delta A\|
        \leq
        \frac{\|A^{-1}\|\|\delta A\|}{1 - \| A^{-1}\|\|\delta A\|}\\
        =&
        \frac{\mathrm{cond}(A)\frac{\|\delta A\|}{\|A\|}}
        {1 - \mathrm{cond}(A)\frac{\|\delta A\|}{\|A\|}}
    \end{aligned}
\end{equation}

\subsection*{3.7(2-19)}

可以直接解得:
$$
x+\delta x = \frac{\beta + 1}{\alpha + 1}A^{-1}b = \frac{\beta + 1}{\alpha + 1}x
\Rightarrow \delta x = \frac{\beta - \alpha }{\alpha + 1} x
$$
因此
$$
\frac{\|\delta x\|}{\|x\|}  = \frac{|\beta - \alpha |}{\alpha + 1} 
\leq \frac{|\beta|+|\alpha|}{1-|\alpha|}
$$

\subsection*{3.8(2-21)}

考虑$A^{-1}B$是奇异的,则存在$x$使得$(I-A^{-1}B)x=x$,因此$(I-A^{-1}B)$有1特征值,因此其谱半径
不小于1,因此$\|(I-A^{-1}B)\|\geq 1$

因此
$$
1\leq\|A^{-1}(A-B)\|\leq\|A^{-1}\|\|A-B\|\Rightarrow
\frac{1}{\|A^{-1}\|\|A\|}\leq\frac{\|A-B\|}{\|A\|}\Rightarrow
\frac{1}{\mathrm{cond}(A)}\leq\frac{\|A-B\|}{\|A\|}
$$



% \subsection*{2.1(1-8)}

% 以下认为讨论的数域是$\mathbb{R}$。

% \subsubsection*{(1)}
% 分别满足:

% 线性性:
% 显然
% $$
% \int_a^b{f(h+g) dx}=\int_a^b{fh dx}+\int_a^b{fg dx}
% $$

% 数乘:
% 显然
% $$
% \int_a^b{(\alpha f)g dx}=\alpha\int_a^b{fg dx}
% $$

% 实数域的对称性:
% 显然
% $$
% \int_a^b{fg dx}=\int_a^b{gf dx}
% $$

% 正定性:
% 显然
% $$
% \int_a^b{ff dx}\geq 0
% $$,
% 且由于$ff\geq 0$,若$\av{f,f}>0$,存在一点使得$ff > 0$,
% 因此$f\neq 0$;若$\av{f,f}=0$,
% 则一定有$ff=0$,因此$f=0$。所以满足0元素等价于自内积为0。

% 因此,是内积。

% \subsubsection*{(2)}
% 不是。当$f$在$x_i, i=1,2,\dots m$点上为0,且存在非零区间时,
% 有$f\neq0$且$\av{f,f}=0$,不满足正定性。

% \subsection*{2.2(1-11)}

% \subsubsection*{(1)}
% \begin{equation*}
%     \begin{aligned}
%         \left\|\bm{A}\right\|_1 =& 6\\
%         \left\|\bm{A}\right\|_2 =& \sqrt{\rho\left(
%             \begin{bmatrix}
%                 10 & -14\\-14 & 20
%             \end{bmatrix}
%         \right)}\approx5.4650\\
%         \left\|\bm{A}\right\|_F =& \sqrt{30}\\
%         \rho(\bm{A})\approx&5.372
%     \end{aligned}
% \end{equation*}

% \subsubsection*{(2)}
% \begin{equation*}
%     \begin{aligned}
%         \left\|\bm{A}\right\|_1 =& 4\\
%         \left\|\bm{A}\right\|_2 =& \sqrt{\rho\left(
%             \begin{bmatrix}
%                 5 & -4 & 1\\-4 & 6 & -4\\1 & -4 & 5
%             \end{bmatrix}
%         \right)}\approx3.4142\\
%         \left\|\bm{A}\right\|_F =& 4\\
%         \rho(\bm{A})\approx&3.4142
%     \end{aligned}
% \end{equation*}

% \subsection*{2.3(1-12)}
% \subsubsection*{(1)}
% $$
% \|\bm{x}\|_\infty = \max{\{|x_i|\}}\leq\sum{|x_i|} = \|\bm{x}\|_1
% $$
% 左侧得证。
% $$
% \|\bm{x}\|_1=\sum{|x_i|}\leq\sum_{i=1}^n{\max{\{|x_i|\}}}=n\max{\{|x_i|\}}
% =n\|\bm{x}\|_\infty
% $$
% 右侧得证。

% \subsubsection*{(2)}
% $$
% \|\bm{x}\|_\infty = \sqrt{(\max{\{|x_i|\}})^2}
% =\sqrt{\max{\{x_i^2\}}}\leq\sqrt{\sum{x_i^2}}=\|\bm{x}\|_2
% $$
% 左侧得证。
% $$
% \|\bm{x}\|_2=\sqrt{\sum{x_i^2}}\leq\sqrt{\sum{\max{\{x_i^2\}}}}
% =\sqrt{n}\sqrt{(\max{\{|x_i|\}})^2}=\sqrt{n}\|\bm{x}\|_\infty
% $$
% 右侧得证。

% \subsubsection*{(3)}
% 取二范数下单位矢量$\bm{x}$。

% 根据从属范数的定义,存在$\bm{x}$使得$\|\bm{Ax}\|_2=\|\bm{A}\|_2$,
% 记这个单位向量是$\bm{x_1}$。
% 已经证明F范数与向量2范数相容,因此
% $$
% \|\bm{A}\|_2=\|\bm{Ax_1}\|_2\leq\|\bm{A}\|_F
% $$
% 左侧得证。

% 设$\bm{A}$列向量是$\bm{a_i}$
% 因此有:
% $$
% \|\bm{A}\|_F^2=\sum_i{\|\bm{a_i}\|_2^2}
% \leq n \max{\{\|\bm{a_i}\|_2^2\}}
% =n\|\bm{A}\bm{e}_{\mathrm{argmax}\{\|\bm{a_i}\|_2^2\}}\|_2^2
% \leq n\|\bm{A}\bm{x_1}\|_2^2
% =n\|\bm{A}\|_2^2
% $$
% 其中$\mathrm{argmax}$指的是取最大值时的下标。
% 右侧得证。

% \subsection*{2.4(1-15)}

% 正定阵进行特征分解:$\bm{A}=\bm{Q}\trans\bm{DQ}$,$\bm{D}$
% 为正值对角阵。

% 则:

% 正定性:$(\bm{Ax},\bm{x})=\bm{x}\trans\bm{Q}\trans\bm{DQx}$
% 容易发现,由于对角阵为正对角,设$\bm{Qx}=\bm{y}=[y_1,y_2,...]\trans$,
% 因此有$(\bm{Ax},\bm{x})=\lambda_1y_1^2+\lambda_2y_2^2...$,因此
% 可知$(\bm{Ax},\bm{x})\geq 0 $且
% $(\bm{Ax},\bm{x})=0 \Leftrightarrow \bm{y}=0 $,
% 又由于$\bm{Q}$是可逆的,$\bm{y}=0\Leftrightarrow\bm{x}=0$,
% 因此内积正定性满足,开根号后满足范数正定性。


% 齐次性:
% 根据内积的线性性,和对称性,
% 可知$(\bm{A}(\alpha \bm{x}),\alpha\bm{x})=\alpha^2(\bm{Ax},\bm{x})$
% ,开根号后满足齐次性。

% 三角不等式:
% $$
% \|\bm{x+y}\|^2_A=\|\bm{x}\|^2_A+\|\bm{y}\|^2_A+2(\bm{Ax},\bm{y})
% \leq\|\bm{x}\|^2_A+\|\bm{y}\|^2_A+2\|\bm{y}\|_A\|\bm{x}\|_A
% =(\|\bm{y}\|_A+\|\bm{x}\|_A)^2
% $$
% 上式中间为柯西-施瓦茨不等式。
% 因此满足三角不等式。

% 综上为范数。

% \subsection*{2.5(1-17)}

% 由于两个矩阵的任意性,以及正交阵的转置还是正交阵,下面证明$\bm{QA}$相关
% 的等号就证明了$\bm{AQ}$的等号。

% 根据二范数的生成范数定义,找到使得$\|\bm{Ax}\|_2$最大的单位向量$\bm{x}$
% 后,由于左乘正交阵$\bm{Q}$不改变$\|\bm{Ax}\|$的二范数,
% 可知取最大值的$\bm{x}$不变,则对应的矩阵二范数不变,第一个式子证明。

% 设$A$的列向量是$\bm{a_i}$,由于
% $\|\bm{A}\|_F^2=\sum_i\|\bm{a_i}\|_2^2$,且
% $\|\bm{QA}\|_F^2=\sum_i\|\bm{Qa_i}\|_2^2$
% ,又正交变换不改变列向量的二范数,所以F范数也没有改变。

% \subsection*{2.6(1-19)}
% \subsubsection*{(1)}
% 考虑乘积矩阵的元素:
% $$
% c_{ik}=\sum_{j}a_{ij}b_{jk}
% $$
% 当$i<k$,对于所有的$j$,都有$j<k$或者$j>i$,
% 因此$a_{ij}=0$或者$b_{jk}=0$,因此求和中每一项是0,因此$c_{ik}=0$。

% \subsubsection*{(2)}
% 假如对角线有一个元素$a_{mm}=0$,容易发现,$i=m,m+1,...$这部分$n-m+1$个
% 列向量构成的矩阵只有$n-m$行是非零的,
% 因此这部分列向量是线性相关的,因此矩阵不可逆。

% 因此,可逆的下三角阵一定是对角全非零。考虑:

% $$
% \bm{A}\bm{A}^{-1}=\bm{I}
% $$

% 通过初等矩阵左乘将$A$变换成对角阵的过程中,左乘的矩阵操作是将某一行之后的
% 所有行减去这一行的倍数,具体是$\bm{L}_j$:

% $$
% \bm{l}_j=(0,0,...,-a_{j+1,j}/a_{jj},...-a_{n,j}/a{jj}),
% \bm{L}_j=I+\bm{l}_j\bm{e}_j\trans
% $$

% 其中高斯消元中由于下三角性质前面的操作不影响后面操作的矩阵元素,因此
% 实际上第$j$步高斯消元的矩阵元素都是最初的值。

% 则
% $$
% \bm{D}\bm{A}^{-1} = \bm{L}_n\dots\bm{L}_2\bm{L}_1
% $$
% $\bm{D}$是消元后的对角阵。
% 其中每个初等矩阵都是下三角的,所以逆矩阵也是下三角的。







\end{document}