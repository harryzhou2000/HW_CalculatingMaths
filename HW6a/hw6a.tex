%!TEX program = xelatex
\documentclass[UTF8,zihao=5]{ctexart}


\title{{\bfseries 第6次作业}}
\author{周涵宇 2022310984}
\date{}

\usepackage[a4paper]{geometry}
\geometry{left=0.75in,right=0.75in,top=1in,bottom=1in}

\usepackage[
UseMSWordMultipleLineSpacing,
MSWordLineSpacingMultiple=1.5
]{zhlineskip}

\usepackage{fontspec}
\setmainfont{Times New Roman}
% \setmonofont{JetBrains Mono}
\setCJKmainfont{仿宋}[AutoFakeBold=true]
\setCJKsansfont{黑体}[AutoFakeBold=true]

% \usepackage{bm}

\usepackage{amsmath,amsfonts}
\usepackage{array}
\usepackage{float}

\newcommand{\bm}[1]{{\mathbf{#1}}}

\newcommand{\trans}[0]{^\mathrm{T}}
\newcommand{\tran}[1]{#1^\mathrm{T}}
\newcommand{\hermi}[0]{^\mathrm{H}}

\newcommand*{\av}[1]{\left\langle{#1}\right\rangle}

\newcommand*{\avld}[1]{\frac{\overline{D}#1}{Dt}}

\newcommand*{\pd}[2]{\frac{\partial #1}{\partial #2}}

\newcommand*{\pdcd}[3]
{\frac{\partial^2 #1}{\partial #2 \partial #3}}


\begin{document}

\maketitle

\subsection*{6.1(6-2)}

有插值多项式节点(拉格朗日)基为:

$$
    l_i(x) = \prod_{j=0,j\neq i}^n{\frac{
        x-x_j
    }{
        x_i-x_j
    }}
$$

则三个节点时有插值多项式:

$$
    L_2=
    f(x_0)\frac{(x-x_1)(x-x_2)}{(x_0-x_1)(x_0-x_2)}+
    f(x_1)\frac{(x-x_2)(x-x_0)}{(x_1-x_2)(x_1-x_0)}+
    f(x_2)\frac{(x-x_0)(x-x_1)}{(x_2-x_0)(x_2-x_1)}
$$

$$
    x_0=1,~x_1=1.05,~x_2=1.07
$$
$$
    [f(x_0),f(x_1),f(x_2)]\approx[2.718281828459046,3.286298785772639,3.527609194972166]
$$

计算可得具体的数值,$L(1.03)=3.053047591586878$,
绝对误差$1.1417e-04$。

误差余项为

$$
    R_2(x)=\frac{f^{(3)}(\xi)}{3!}\omega_3(x)
    =\frac{7e^{\xi}+3\xi e^{\xi}}{3!}(x-1)(x-1.05)(x-1.07)
$$

其中$\xi\in(1,1.07)$

则由单调性,

$$
    |R_2(1.03)|<|\frac{f^{(1.07)}(\xi)}{3!}\omega(1.03)|
    = 1.1906e-04
$$

可见这个误差估计比较精确。

\subsection*{6.2(6-12)}
不妨采用牛顿插值:

均差表是:
\begin{table}[H]
    \centering
    \begin{tabular}{c|cccc}
        x                       \\
        \hline
        0 & 0 &   &      &      \\
        0 & 0 & 0 &      &      \\
        1 & 1 & 1 & 1    &      \\
        2 & 1 & 0 & -1/2 & -3/4 \\
    \end{tabular}
\end{table}

那么,可知插值函数

$$
    N_3(x)=0 + 0(x-0) + 1(x-0)(x-0) - \frac{3}{4}(x-0)(x-0)(x-1)
    =\frac{7}{4}x^2-\frac{3}{4}x^3
$$

余项
$$
    R_3(x)=\omega_3(x)f[x,0,0,1,2]
    =f[x,0,0,1,2]x^2(x-1)(x-2)
$$


\subsection*{6.3(6-13)}
不妨采用牛顿插值:

均差表是:
\begin{table}[H]
    \centering
    \begin{tabular}{c|ccccc}
        x                        \\
        \hline
        1 & 0 &    &    &    &   \\
        1 & 0 & 0  &    &    &   \\
        1 & 0 & 0  & 4  &    &   \\
        2 & 1 & 1  & 1  & -3 &   \\
        3 & 0 & -1 & -1 & -1 & 1 \\
    \end{tabular}
\end{table}

那么,可知插值函数

$$
    N_4(x)=4(x-1)^2-3(x-1)^3+(x-1)^3(x-2)
$$

余项
$$
    R_4(x)=f[x,1,1,1,2,3](x-1)^3(x-2)(x-3)
$$


\subsection*{6.4(6-16)}

$$
    S'(x)=\left\{
    \begin{array}{lr}
        2-3x^2,              & x\in(0,1) \\
        b+2c(x-1)+3d(x-1)^2, & x\in(1,2) \\
    \end{array}
    \right.
$$

$$
    S''(x)=\left\{
    \begin{array}{lr}
        -6x,        & x\in(0,1) \\
        2c+6d(x-1), & x\in(1,2) \\
    \end{array}
    \right.
$$

$x=1$处1阶导数连续可知:$-1=b$

$x=1$处2阶导数连续可知:$-6=2c$

$x=2$处2阶导数给定可知:$2c+6d=0$

综上解得$c=-3,~b=-1,~d=1$

\subsection*{6.5(6-17)}

$$
    S'(x)=\left\{
    \begin{array}{lr}
        B+4x-6x^2,          & x\in(0,1) \\
        b-8(x-1)+21(x-1)^2, & x\in(1,2) \\
    \end{array}
    \right.
$$

$x=1$处0阶导数连续可知:$B+1=1$

$x=1$处1阶导数连续可知:$B-2=b$

解得$b=-2,~B=0$,可知$S'(0)=B=0,~S'(2)=13+b=11$

\subsection*{6.6(6-19)}

\subsubsection*{(1)}
设出节点的一阶导数$m_i$,已知$m_0=-1,~m_1=1$
则需要求解$m_1,m_2$,
根据$x_1,x_2$的二阶导数连续要求,以及Hermite插值
基函数的二阶导数,列方程有:

$$
    \begin{bmatrix}
        2   & 1/4 \\
        1/2 & 2   \\
    \end{bmatrix}\begin{bmatrix}
        m_1 \\
        m_2 \\
    \end{bmatrix}
    =\begin{bmatrix}
        -3 \\
        0  \\
    \end{bmatrix}
$$

其中用到了$[\lambda_1,\lambda_2]=[1/4,1/2]$,
$[\mu_1,\mu_2]=[3/4,1/2]$,
$[d_1,d_2]=[-15/4,1/2]$

解得:$m_1=-48/31~m_2=12/31$,因此
代入分段Hermite插值有:

$$
    S(x)=\left\{
    \begin{array}{lr}
        \frac{500}{31}+\frac{728\,x}{31}+\frac{329\,x^2}{31}+\frac{45\,x^3}{31},       & x\in(-3,-2) \\
        \frac{476}{279}+\frac{172\,x}{93}-\frac{19\,x^2}{93}-\frac{98\,x^3}{279},      & x\in(-2,1)  \\
        \frac{881}{837}+\frac{1063\,x}{279}-\frac{604\,x^2}{279}+\frac{253\,x^3}{837}, & x\in(1,4)   \\
    \end{array}
    \right.
$$

\subsubsection*{(2)}
设出节点的一阶导数$m_i$,边界上给出二阶导数条件两个,
根据$x_1,x_2$的二阶导数连续要求,以及Hermite插值
基函数的二阶导数,列方程有:

$$
    \left[\begin{array}{cccc} 2 & 1 & 0 & 0\\ \frac{3}{4} & 2 & \frac{1}{4} & 0\\ 0 & \frac{1}{2} & 2 & \frac{1}{2}\\ 0 & 0 & 1 & 2 \end{array}\right]
    \begin{bmatrix}
        m_0 \\
        m_1 \\
        m_2 \\
        m_3 \\
    \end{bmatrix}
    =\left[\begin{array}{c} -6\\ -\frac{15}{4}\\ \frac{1}{2}\\ -2 \end{array}\right]
$$

其中用到了$[\lambda_1,\lambda_2]=[1/4,1/2]$,
$[\mu_1,\mu_2]=[3/4,1/2]$,
$[d_1,d_2]=[-15/4,1/2]$

解为$\left[\begin{array}{cccc} -\frac{215}{87} & -\frac{92}{87} & \frac{76}{87} & -\frac{125}{87} \end{array}\right]\trans  $
,因此
代入分段Hermite插值有:
$$
    S(x)=\left\{
    \begin{array}{lr}
        \frac{212}{29}+\frac{892\,x}{87}+\frac{123\,x^2}{29}+\frac{41\,x^3}{87},      & x\in(-3,-2) \\
        \frac{1252}{783}+\frac{440\,x}{261}-\frac{11\,x^2}{261}-\frac{190\,x^3}{783}, & x\in(-2,1)  \\
        \frac{995}{783}+\frac{697\,x}{261}-\frac{268\,x^2}{261}+\frac{67\,x^3}{783},  & x\in(1,4)   \\
    \end{array}
    \right.
$$


\end{document}