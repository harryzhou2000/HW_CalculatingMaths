%!TEX program = xelatex
\documentclass[UTF8,zihao=5]{ctexart}


\title{{\bfseries 第1次作业}}
\author{周涵宇 2022310984}
\date{}

\usepackage[a4paper]{geometry}
\geometry{left=0.75in,right=0.75in,top=1in,bottom=1in}

\usepackage[
UseMSWordMultipleLineSpacing,
MSWordLineSpacingMultiple=1.5
]{zhlineskip}

\usepackage{fontspec}
\setmainfont{Times New Roman}
% \setmonofont{JetBrains Mono}
\setCJKmainfont{仿宋}[AutoFakeBold=true]
\setCJKsansfont{黑体}[AutoFakeBold=true]

% \usepackage{bm}

\usepackage{amsmath,amsfonts}
\usepackage{array}
\usepackage{float}

\newcommand{\bm}[1]{{\mathbf{#1}}}

\newcommand{\trans}[0]{^\mathrm{T}}
\newcommand{\tran}[1]{#1^\mathrm{T}}
\newcommand{\hermi}[0]{^\mathrm{H}}

\newcommand*{\av}[1]{\left\langle{#1}\right\rangle}

\newcommand*{\avld}[1]{\frac{\overline{D}#1}{Dt}}

\newcommand*{\pd}[2]{\frac{\partial #1}{\partial #2}}

\newcommand*{\pdcd}[3]
{\frac{\partial^2 #1}{\partial #2 \partial #3}}


\begin{document}

\maketitle

\subsection*{1.1(1-2)}

记$y=1-\cos{2^\circ}$,则有以下计算:

\subsubsection*{(1)}

$y_1=6\times10^{-4}$,
实际误差:$e_1\approx9.2\times10^{-6}, e_{r1}\approx0.015$
有一位有效数字。

\subsubsection*{(2)}

$y_2=6.125\times10^{-4}$,
实际误差:$e_2\approx3.3\times10^{-6}, e_{r2}\approx0.0055$
有两位有效数字。

\subsubsection*{(3)}

$y_3\approx6.0918776\times10^{-4}$,
实际误差:$e_3\approx1.5\times10^{-8}, e_{r3}\approx2.4\times10^{-5}$
有四位有效数字。

\subsubsection*{(4)}

\begin{equation*}
    1-\cos{x} = \frac{1}{2}x^2-\frac{1}{24}x^4+\frac{\sin{\theta}}{120}x^5,\ \
    \mathrm{where}\ \theta\in\left[0,x\right]
\end{equation*}

上式可知,代入$x=\frac{\pi}{90}$,则误差:
$e_4<\frac{x^6}{120}<1.6\times10^{-11}$,
$e_{r4}<\frac{x^6}{120\times6\times10^{-4}}<2.5\times10^{-8}$
其中绝对误差小于$0.5\times10^{-8}$,因此至少有四位有效数字:

\begin{equation*}
    y_4=\frac{1}{2}(\frac{\pi}{90})^2-\frac{1}{24}(\frac{\pi}{90})^4\approx6.0917298\times10^{-4}
\end{equation*}
至少八位有效数字。

\subsection*{1.2(1-3)}

考虑两个公式都是$e^x$在$0$展开的结果,因此根据Lagrange余项,易得:

记$y_1=\sum_{n=0}^9{(-1)^n\frac{5^n}{n!}}$,
有$\left|e^{-5}-y_1\right|<\frac{5^{10}}{10!}$

记$y_2=\left(\sum_{n=0}^9{\frac{5^n}{n!}}\right)^{-1}$,
有$\left|e^{5}-y_2^{-1}\right|<\frac{5^{10}}{10!}$则有,
则可知$\left|e^{-5}-y_2\right|<\frac{5^{10}}{10!e^{10}}
    +O\left(\left(\frac{5^{10}}{10!}\right)^2\right)<\frac{5^{10}}{10!}$

可知第二种算法误差小,误差限相差大约$e^{10}$。

\subsection*{1.3(1-5)}

\subsubsection*{(1)}

原式是两个1量级的数相减得到很小的数,因此进行三角函数变换:

\begin{equation*}
    \text{原式}=\arctan{\frac{N+1-N}{1+N(N+1)}}=\arctan{\frac{1}{1+N(N+1)}}
\end{equation*}

这样,相对误差意义上,只要三角函数计算足够准确,就不会有明显的精度损失。

\subsubsection*{(2)}

同样,是相减中损失相对误差,恰好可以化为:

\begin{equation*}
    \text{原式}=\frac{1}{\sqrt{x+\frac{1}{x}}+\sqrt{x-\frac{1}{x}}}\frac{2}{x}
\end{equation*}

这样,没有出现大数相减得到小数,相对误差较小。

\subsubsection*{(3)}

\begin{equation*}
    \text{原式}=\ln{\frac{x+1}{x}}
\end{equation*}

只要对数函数计算足够准确,相对误差较小。

\subsubsection*{(4)}

\begin{equation*}
    \text{原式}=\cos{2x}
\end{equation*}

只要三角函数计算足够准确,相对误差较小。

\subsection*{1.4(1-7)}

\subsubsection*{(1)}
计算可得(四舍五入),

\begin{table}[H]
    \begin{center}
        \begin{tabular}{lll}
            $n$ & $I_n$(双精度有效部分) & $\widetilde{I_n}$ \\
            \hline
            0   & 0.632120558828558     & 0.6321        \\
            1   & 0.367879441171442     & 0.3679        \\
            2   & 0.264241117657115     & 0.2642        \\
            3   & 0.207276647028654     & 0.2074        \\
            4   & 0.170893411885384     & 0.1704        \\
            5   & 0.145532940573080     & 0.1480        \\
            6   & 0.126802356561520     & 0.1120        \\
            7   & 0.112383504069363     & 0.2160        \\
            8   & 0.100931967445092     & -0.7280       \\
            9   & 0.0916122929941707    & 7.552         \\
            \hline
        \end{tabular}
    \end{center}
\end{table}

近似结果是7.552,误差很大。

\subsubsection*{(2)}
每次计算的$1$和$n$都是准确表示,则容易通过递推
$\varepsilon_n=I_n-\widetilde{I_n}=n\left(I_{n-1}-\widetilde{I_{n-1}}\right)$发现,
$|\varepsilon_n|=n!|\varepsilon_0|$
考虑每次运算舍入误差的话,每次增加误差限$\varepsilon_{mr}\widetilde{I_n}$,则
$|\varepsilon_n|\leq n!|\varepsilon_0|+\sum_{k=1}^n{(n-k)!\left|\varepsilon_{mr}\widetilde{I_n}\right|}$


\end{document}