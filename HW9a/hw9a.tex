%!TEX program = xelatex
\documentclass[UTF8,zihao=5]{ctexart}


\title{{\bfseries 第9次作业}}
\author{周涵宇 2022310984}
\date{}

\usepackage[a4paper]{geometry}
\geometry{left=0.75in,right=0.75in,top=1in,bottom=1in}

\usepackage[
UseMSWordMultipleLineSpacing,
MSWordLineSpacingMultiple=1.5
]{zhlineskip}

\usepackage{fontspec}
\setmainfont{Times New Roman}
% \setmonofont{JetBrains Mono}
\setCJKmainfont{仿宋}[AutoFakeBold=true]
\setCJKsansfont{黑体}[AutoFakeBold=true]

% \usepackage{bm}

\usepackage{amsmath,amsfonts}
\usepackage{array}
\usepackage{float}

\newcommand{\bm}[1]{{\mathbf{#1}}}

\newcommand{\trans}[0]{^\mathrm{T}}
\newcommand{\tran}[1]{#1^\mathrm{T}}
\newcommand{\hermi}[0]{^\mathrm{H}}

\newcommand*{\av}[1]{\left\langle{#1}\right\rangle}

\newcommand*{\avld}[1]{\frac{\overline{D}#1}{Dt}}

\newcommand*{\pd}[2]{\frac{\partial #1}{\partial #2}}

\newcommand*{\pdcd}[3]
{\frac{\partial^2 #1}{\partial #2 \partial #3}}


\begin{document}

\maketitle

\subsection*{9.1(9-5)}

$$
T(x,y,h) = hy' + \frac{h^2}{2}y'' + O(h^3) - hf(x+h,y+hf)
$$
其中默认指$x_n,y_n$位置的值。考虑
$$
f(x+h,y+hf)=f+\pd{f}{x}h+\pd{f}{y}hf+O(h^2)
=y'+y''h+O(h^2)
$$
代入有
$$
T(x,y,h) = -\frac{h^2}{2}y'' + O(h^3)
$$
可见总体一阶精度,误差主项是$-\frac{h^2}{2}y''$。

代入$f=\lambda y$,代入有:
$$
y_{n+1}=y_n(1+h\lambda+h^2\lambda^2)
$$
使$|1+h\lambda+h^2\lambda^2|<1$的区间为$(-1,0)$,为绝对稳定性区间。

\subsection*{9.2(9-8)}

$$
T(x,y,h) = hy' + \frac{h^2}{2}y'' + \frac{h^3}{6}y''' + O(h^4)
-hf(x+\frac{h}{2},y+\frac{h}{2}f)
$$
其中默认指$x_n,y_n$位置的值。考虑
$$
f(x+\frac{h}{2},y+\frac{h}{2}f)=
f
+\frac{h}{2}\pd{f}{x}
+\frac{hf}{2}\pd{f}{y}
+\frac{h}{8}\pd{^2f}{x^2}
+\frac{h^2f^2}{8}\pd{^2f}{y^2}
+\frac{2h^2f}{8}\pdcd{f}{x}{y}
+O(h^3)
$$
代入即有
$$
T(x,y,h)
=
h^3\left(
    \frac{y'''}{6}-\frac{\pd{^2f}{x^2}+\pd{^2f}{y^2}f^2+\pdcd{f}{x}{y}2f}{8}
    \right)
    +O(h^4)
$$
如果考虑
$$
\begin{aligned}
    y'''& =\pd{^2f}{x^2}+\pdcd{f}{x}{y}f+
    f\left(
        \pd{^2f}{y^2}f+\pdcd{f}{x}{y}
    \right)
    +
    \left(
        \pd{f}{x}+\pd{f}{y}f
    \right)\pd{f}{y}\\
    & = \pd{^2f}{x^2} + 2\pdcd{f}{x}{y}f + \pd{^2f}{y^2}f^2
    +\pd{f}{x}\pd{f}{y}+\left(\pd{f}{y}\right)^2f
\end{aligned}
$$
则截断误差主项有:
$$
Tp = h^3\left(
    \frac{
        \pd{^2f}{x^2} + 2\pdcd{f}{x}{y}f + \pd{^2f}{y^2}f^2}{24}
    +\frac{
        \pd{f}{x}\pd{f}{y}+\left(\pd{f}{y}\right)^2f}{6}
    \right)
$$
可见是二阶格式。

\subsection*{9.3(9-10)}

将试验方程$y'=\lambda y$代入,有
$$
y_{n+1}=y_n+h\lambda\frac{y_n+y_{n+1}}{2}
$$
解得
$$
y_{n+1}=\frac{1+\frac{h\lambda}{2}}{1-\frac{h\lambda}{2}}y_n
$$
当$h\lambda\in\mathbb{R}$,即可知
$\left|\frac{1+\frac{h\lambda}{2}}{1-\frac{h\lambda}{2}}\right|<1\Leftrightarrow h\lambda<0$。
因此绝对稳定性区间$(-\infty,0)$。

\subsection*{9.4(9-16)}

\subsection*{9.5(9-17)}

\subsection*{9.6(9-18)}




\end{document}